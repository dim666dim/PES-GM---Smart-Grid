\section{Problem formulation and mdp model} \label{sec:model}
%Microgrids comprise of the distributed small scale power generating sources, mainly renewable energy sources (mainly wind and solar ) and  are also equipped with storage devises.
We consider $N$ microgrids denoted by $\{1,\ldots,N\}$, which are inter-connected through distribution network. In this paper, we consider the case when microgrids are connected  to the main grid (i.e, they are operated in grid connected mode). Each microgrid comprise of the distributed small scale renewable power generating sources and also equipped with energy storage devices. Let $B_i$ be the maximum energy storage capacity of microgrid $i$. At every time step $t$ of a day, $i$th microgrid controller $C_i$ is having the following information:
\begin{enumerate}[label=(\alph*)]
\item Total generated energy from all its distributed renewable energy sources denoted by  $r_t^i$.
% (super script $i$ used to refer to microgrid $i$).
\item Accumulated non-ADL demand denoted by $d_t^i$, from each load in the $i$th microgrid. 
\item Set of all ADL jobs at microgrid $i$ denoted $J_{t}^{i}$.  $J_{t}^{i}$ is of the form $\{\gamma_{1}^{i},\ldots,\gamma_{n}^{i}\}$, where $j$th ADL job $\gamma_{j}^{i}$, is a tuple consists of  number of units of energy required to finish that job denoted by $a_{j}^{i}$ and an integer  $f_{j}^{i}$, which denotes number of future time slots remaining by which one can schedule that job without incurring the penalty. Let  it be represented as follows $\gamma_{j}^{i} = (a_{j}^{i}, f_{j}^{i})$. Let the total ADL demand be denoted by $A_t^i= \sum_{j=1}^{n} a_j^i$.
\item  Total energy available in the storage device of microgrid $i$, denoted by $b_{t}^{i}$.
\end{enumerate} 
In this paper, we are considering the cooperative energy exchange model under which microgrids can share energy among themselves. 
From the above available information, microgrid controller  $C_i$ at every time step $t$ has to decide on the following choices: (a)  Amount of energy it needs to buy/sell from the main grid, (b) Amount of energy it needs to buy/sell from the neighboring microgrids,
(c) Amount of the energy it needs to store/take from the storage device, and (d) Sub-set of ADL jobs it needs to schedule. Both the demand and energy generated at microgrid $i$ is uncertain/random due to  random nature of  loads ($d_t^i$ and $A_t^i$) and renewable energy generation ($r_t^i$). 

Markov decision process (MDP)  is a general framework for modeling problems of dynamic optimal decision making under uncertainty. *****Need to write about MDP*******.
%We modeled our problem in the framework of MDP. 
 In the next sub-section we provide the details of our MDP model.
\subsection{MDP framework}
\subsubsection{State space}
The state $s_{t}^{i}$ at time instant $t$  for microgrid $i$ is as follows:
\begin{align}
s_{t}^{i} = (t,nd_{t}^{i},p_{t}, J_{t}^{i}),
\end{align}
where the net demand $nd_{t}^{i} = r_{t}^{i} + b_{t}^{i} - d_{t}^{i}$, which denotes whether there is an excess power or deficit.  And $p_{t} and  J_{t}^{i}$ denotes the price per unit energy and set of all ADL jobs at time instant $t$ respectively. If  net demand $nd_{t}^{i}$ is positive then it implies that there is excess of power after meeting the non-ADL demand and if negative implies that there is a deficit in power even to meet the non-ADL demand. Current time slot is  included in the state since optimal action can depend on time. For example, optimal action for solar microgrid  is to sell the power in the morning as it can generate enough power in the afternoon. But it is not optimal to sell in the evening as it can not generate power in the night. Optimal action in the morning is to sell whereas in the evening is store.
\subsubsection{Action space}
At each time instant $t$ microgrid controller needs to take two decision $u_{t}^{i}$ and $v_{t}^{i}$. The first action $u_{t}^{i}$, if positive, denotes the number of units that the microgrid is willing to sell and if negative, represents the number of units that the microgrid is willing to buy. The second action $V_{t}^{i}$ pertains to the scheduling decision of ADL jobs taken by microgrid $i$.

Let $P_{t}^{i}$ be the power set of $J_{t}^{i}$, which consists of all possible combinations of the ADL jobs that can be scheduled at time instant $t$ at microgrid $i$. Let  $A_{t}^{i}$ consists of the total aggregated demand  for every subset in  $P_{t}^{i}$. For example, let the $j$th element of $A_{t}^{i}(j) = \sum_{k=1, \gamma_k^i \in P_{t}^{i}(j) }^n a_k^i$, where $ P_{t}^{i}(j)$ is $j$th element of  $P_{t}^{i}$.
%Let $P_{t}^{i} = \{\Gamma_{1}^{i},\ldots,\Gamma_{N}^{i}\}$ be the power set of $J_{t}^{i}$, which consists of all possible combinations of the ADL jobs that can be sheduled at time instant $t$ at microgrid $i$. 
%Let  $A_{t}^{i} = \{A(\Gamma_{1}^{i}),\ldots,A(\Gamma_{N}^{i})\} $, where $A(\Gamma_{j}^{i}) = \sum_{\gamma_{k}^{i} \in \Gamma_{j}^{i} } a_{k}^{i}$.
 The feasible region for action $u_{t}^{i}$ is bounded as follows:
\begin{align}
-min(M_t^i, B_t^i - nd_t^i + &\max_{1\leq j \leq 2^n} A(j) ) \leq u_t^i \nonumber\\ &\leq max(0, nd_t^i - \min_{1\leq j \leq 2^n} A(j)),
\end{align}
where $M_t^i$ denotes maximum amount of power main grid can give it to microgrid $i$. This constraint is to maintain stability of the main grid. Above bounds represents the following, microgrids can sell the power only after meeting its non-ADL demand and can buy to fill its battery after meeting its both ADL and non-ADL demands.
%The intuition behind the bounds is as follows. A microgrid can sell atmost the excess power. That is, the power remaining after meeting the demand. While buying, it can buy to meet the demand and also to fill its battery.

After controller picks action $u_{t}^{i}$, we construct the feasible set $F_{t}^{i}$, which is a subset of $P_{t}^{i}$. It consists of all possible subsets of ADL jobs that can be scheduled with $u_{t}^{i}$. More formally, each element $j$ of  $F_{t}^{i}$ has to satisfy the following condition :   $A_t^i(j) \leq u_{t}^{i} $, where $A_t^i(j)$ is total energy required to finish all the ADL jobs in it. Now, controller picks action $v_{t}^{i}$ which is an element in $F_{t}^{i}$, which results in scheduling all the ADL jobs in that subset. The remaining power is used to meet the non-ADL demand and for storing in the battery for future use.

%Each element $\Gamma_{j}^{i}$ in the $F_{t}^{i}$ has to satisfy the following condition $A(\Gamma_{j}^{i}) \leq u_{t}^{i} $.
%Agent has to pick the action $v_{t}^{i} = \Gamma_{j}^{i} \in F_{t}^{i}$. Now ADL jobs in $\Gamma_{j}^{i}$ will get sheduled. 
 Let $\widehat J_{t+1}^{i}$ be the new set of ADL jobs received by controller at time instant $t+1$. Depends on action $v_{t}^{i}$, some of the ADL jobs will not get scheduled. We pass them to time step $t+1$, if they can be scheduled without incurring the penalty. The set of all ADL jobs at time instant $t+1$ is union of the new ADL jobs and old ADL jobs which are not scheduled after reducing there  $f_{j}^{i}$ by one ( number of future time slots remaining by which one can schedule that job without incurring the penalty) .
%For the next time instant $t+1$, we update the following :
$J_{t+1}^{i} = \widehat J_{t+1}^{i} \cup \widetilde J_{t}^{i}$, where $\widetilde J_{t}^{i} =  \{(a_{1}^{i}, f_{1}^{i} - 1),\ldots,(a_{n}^{i}, f_{n}^{i} - 1)\}$, and $ (a_{j}^{i}, y_{j}^{i}) \in \overline J_{t}^{i}$. And $\overline J_{t}^{i} = J_{t}^{i} - v_{t}^{i}$.

The storage device battery information is updated as follows:
\begin{align}
b_{t+1}^{i} = max(0,nd_{t}^{i} - u_{t}^{i}),
\end{align}
which denotes the power available after meeting the non-ADL demand and ADL demand is stored in the battery for future use.
\subsubsection{Single stage reward function}
In this paper, we want to maximize the profit of each microgrid obtained by selling power while reducing the demand and supply deficit. Our singe stage reward function has both the reward obtained by selling power and penalty for unmet demand. The single stage reward  function for our MDP is as follows :
\begin{align}
g_t^{i}(s_t^i,u_t^i) = p_{t}*u_{t}^{i} + c*(min(0,&nd_{t}^{i} - u_{t}^{i}))  \nonumber\\ &+ c* \sum_{k =1}^{n} I_{f_{k}^{i} = 0} a_{k}^{i} ,
\end{align}
The first term represents the cost/gain incurred for  buying/selling the power, and the second and third terms represents the penalty  incurred for not meeting the non ADL demand and ADL demand respectively. Here, $c$ is penalty per unit of unmet demand and $I_{f_{k}^{i}}$ is indicator random variable which is equal to one if $f_{k}^{i} =0$ and zero otherwise. 
%The microgrid incurs a cost of $c$ for every unit of demand that is not met. 
******Need to write about transition probability kernel ***********.
\subsection{Average cost setting}
Finally, the objective of the microgrid $i$ is to maximize the following \cite{avgcost}:
\begin{align}
limsup_{n \rightarrow \infty} 1/n \sum_{k = 0}^{n} E(g^{i}(s_{k},u_{k})),
\end{align}
where $E(.)$ is the expectation. 

We also consider the long run discounted cost formulation. The objective here is to maximize the following:

\begin{align}
limsup_{n \rightarrow \infty} \sum_{k = 0}^{n} \gamma^{k} * E(g^{i}(s_{k},u_{k})),
\end{align}

where $\gamma$ is the discount factor. 

%%%%%%%%%%%%%%%%%%%%%%%%%%%%%%%%%%%%%%%%%%%%%%%%%%%%%%%%%%%%%%%%%%%%%%%%%%%%%%%%%%%%%%%
%%%%%%%%%%%%%%%%%%%%%%%%%%%%%%%%%%%%%%%%%%%%%%%%%%%%%%%%%%%%%%%%%%%%%%%%%%%%%%%%%%%%%%%
%\algblock{PEval}{EndPEval}
%\algnewcommand\algorithmicPEval{\textbf{\em Function evaluation 1}}
% \algnewcommand\algorithmicendPEval{}
%\algrenewtext{PEval}[1]{\algorithmicPEval\ #1}
%\algrenewtext{EndPEval}{\algorithmicendPEval}
%
%\algblock{PEvalPrime}{EndPEvalPrime}
%\algnewcommand\algorithmicPEvalPrime{\textbf{\em Function evaluation 2}}
% \algnewcommand\algorithmicendPEvalPrime{}
%\algrenewtext{PEvalPrime}[1]{\algorithmicPEvalPrime\ #1}
%\algrenewtext{EndPEvalPrime}{\algorithmicendPEvalPrime}
%
%\algblock{PImp}{EndPImp}
%\algnewcommand\algorithmicPImp{\textbf{\em Gradient descent}}
% \algnewcommand\algorithmicendPImp{}
%\algrenewtext{PImp}[1]{\algorithmicPImp\ #1}
%\algrenewtext{EndPImp}{\algorithmicendPImp}
%
%\algtext*{EndPEval}
%\algtext*{EndPEvalPrime}
%\algtext*{EndPImp}
%%%%%%%%%%%%%%%%% alg-custom-block %%%%%%%%%%%%
%\algblock{PEvalPrimeDouble}{EndPEvalPrimeDouble}
%\algnewcommand\algorithmicPEvalPrimeDouble{\textbf{\em Function evaluation 3}}
% \algnewcommand\algorithmicendPEvalPrimeDouble{}
%\algrenewtext{PEvalPrimeDouble}[1]{\algorithmicPEvalPrimeDouble\ #1}
%\algrenewtext{EndPEvalPrimeDouble}{\algorithmicendPEvalPrimeDouble}
%\algtext*{EndPEvalPrimeDouble}
%
%\algblock{PImpNewton}{EndPImpNewton}
%\algnewcommand\algorithmicPImpNewton{\textbf{\em Newton step}}
% \algnewcommand\algorithmicendPImpNewton{}
%\algrenewtext{PImpNewton}[1]{\algorithmicPImpNewton\ #1}
%\algrenewtext{EndPImpNewton}{\algorithmicendPImpNewton}
%
%\algtext*{EndPImpNewton}
%
%%%%%%%%%%%%%%%%%%%%
%
%\begin{algorithm}[t]
%\begin{algorithmic}
%\State {\bf Input:} 
%initial parameter $x_0 \in \R^N$, perturbation constants $\delta_n>0$, step-sizes $\{a_n, b_n\}$, operator $\Upsilon$.
%\For{$n = 0,1,2,\ldots$}	
%	\State Generate $\{d_n^{i}, i=1,\ldots,N\}$, independent of $\{d_m, m=0,1,\ldots,n-1\}$. 
%	\State For any $i=1,\ldots,N$, $d_n^{i}$ is distributed either as an asymmetric Bernoulli (see \eqref{eq:det-proj}) or Uniform $U[-\eta,\eta]$ for some $\eta >0$ (see Remark \ref{remark:unif}). 
%	\PEval
%	    \State Obtain $y_n^+ = f(x_n+\delta_n d_n) + \xi_n^+$.
%  \EndPEval
%	    \PEvalPrime
%	    \State Obtain $y_n^- = f(x_n-\delta_n d_n) + \xi_n^-$.
%	    \EndPEvalPrime
%	    	    \PEvalPrimeDouble
%	    \State Obtain $y_n = f(x_n) + \xi_n$.
%	    \EndPEvalPrimeDouble
%	    \PImpNewton
%		\State Update the parameter and Hessian as follows:
%		\begin{align*}
%		x_{n+1} = & \; x_n - a_n \Upsilon(\overline H_n)^{-1}\widehat\nabla f(x_n), \\
%\overline H_n = &\; (1-b_{n})  \overline H_{n-1} + b_{n} ( \widehat H_n - \widehat \Psi_n),
%\end{align*}
%where $\widehat H_n$ and $\widehat \Psi_n$ are chosen according to \eqref{eq:2rdsa-estimate-ber} and \eqref{eq:psinhat}, respectively. 
%		\EndPImpNewton
%\EndFor
%\State {\bf Return} $x_n.$
%\end{algorithmic}
%\caption{Structure of 2RDSA-IH algorithm.}
%\label{alg:structure}
%\end{algorithm}
